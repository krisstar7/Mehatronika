% Namen teh navodil je predstaviti razvojno ploščo Erasmus+ GEMS Amethyst in podati jasna ter tehnična navodila za začetek uporabe. Dokument je namenjen uporabnikom, ki želijo s ploščo eksperimentirati, razvijati prototipe ali jo uporabiti v pedagoškem procesu.
Razvojna plošča Erasmus+ GEMS Amethyst je kompaktna, dvoslojna tiskanina dimenzij 40×40 mm, zasnovana za uvajanje v svet mikrokrmilniških sistemov. Središče vezja predstavlja mikrokrmilnik ESP32-C3-WROOM-02, ki temelji na energetsko učinkoviti RISC-V arhitekturi in vključuje brezžični povezavi Wi-Fi ter Bluetooth Low Energy, kar omogoča razvoj tako lokalnih kot tudi IoT aplikacij.